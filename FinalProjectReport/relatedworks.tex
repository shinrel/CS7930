\section{Related Works}
So far, there exists many research works on crowdfunding problem. We divide these works into some trends as following:

Researchers have predicted whether the project can be successfully funded or fail. \cite{greenberg2015public} collected 13,000 projects on Kickstarter and extracted 13 features from each one to develope a classifier to predict project success with 68\% accuracy. \cite{Etter:2013} extends the work and show how the temporal amount of money can help improve the accuracy. \cite{Mitra:2014} focused on text features of project pages and show how using phrases features to predict project success.

Another research trend tries to corelate social media activities during running fund raising campaign to project success and proposed solutions for investor recommendation problem. \cite{lu:2014} studied how the amount of money can be  affected by promotional activities on social media like Twitter. \cite{Rakesh:2015} used promoter network on Twitter to show the corelation between the connectivity of project promoters and project success. They also developed backer recommendatin in which potential investors are suggested. \cite{an2014recommending} proposed different ways of recommending investors by using hypothesis-driven analysis of pledging behavior. \cite{althoff2015donor} presented various factor influenced investor retention which allows to identify different groups of investors.

Comparing with the previous research work, we collected largest dataset consisting of more than 150k projects. Our problem is totally different comparing to existed works. That is, we construct statistical models that examine multiple predictive factors toward building two models: (i) one predicts the number of backers will back for the project and (ii) the another predicts the amount of pledged money that the project can receive.
